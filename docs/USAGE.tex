\documentclass[11pt,a4paper,notitlepage]{article}
\usepackage[utf8]{inputenc}
\usepackage[T1]{fontenc}
\usepackage{graphicx}
\usepackage{enumerate}
\usepackage{xcolor}
\definecolor{bg}{rgb}{0.95,0.95,0.95}
\usepackage{ulem}
\usepackage{listings}
\usepackage{vhistory}
\usepackage[parfill]{parskip}

\lstdefinestyle{BashInputStyle}{
  language=bash,
  basicstyle=\small\sffamily,
  commentstyle=\color{black},    
  numberstyle=\tiny\color{black}, 
  keywordstyle=\color{black},       
  extendedchars=true,              
  numbers=none,
  numbersep=3pt,
  frame=tb,
  columns=fullflexible,
  backgroundcolor=\color{yellow!20},
  linewidth=\linewidth,
  breaklines=true,
  breakatwhitespace=false,           
  showspaces=false,
  keepspaces=true,                 
  captionpos=b,                    
  showspaces=false,                
  showstringspaces=false,          
  showtabs=false,                  
  tabsize=2,                       
}

\title{MIR: Installation and Usage}
\author{Ananya Muddukrishna}
\date{ananya@kth.se}

\begin{document}
\maketitle

% Start of the revision history table
\begin{versionhistory}
\vhEntry{1.0}{2014-06-19}{ananya} {Created}
\end{versionhistory}

\section{Introduction}
MIR is a task-based runtime system library.
MIR is written using C99 and scales well for medium-grained task-based programs. 
MIR provides a direct interface for writing task-based programs.
A subset of OpenMP 3.0 tasks is also supported.
MIR is flexible - the user can experiment with different scheduling policies.
Locality-aware scheduling and data distribution support for NUMA systems is also present.
MIR supports extensive performance analysis and profiling features.
Users can quickly solve performance problems using detailed thread-based and task-based performance information provided by MIR.

\section{Requirements}
Building MIR as a basic task-based runtime system requires:
\begin{itemize}
\item Machine with x86 architecture
\item Linux kernel later than January 2012
\item GCC and Binutils
\item Scons - a build system
\end{itemize}

\subsection{Optional requirements}
Extended features of MIR such as profiling and locality-aware scheduling require these additional software:
\begin{itemize}
\item libnuma - for data-distribution and locality-aware scheduling on NUMA systems
\item PAPI - for reading hardware performance counters during thread-based profiling
\item Paraver - for visualizing thread execution traces
\item Python - for executing profiling scripts
\item Intel Pin sources - for profiling instructions executed by tasks during task-based profiling
\item R - for executing profiling scripts
\item These R packages: 
\begin{itemize}
\item igraph - for task graph processing
\item RColorBrewer - for colors
\item gdata, plyr - for data structure transformations
\end{itemize}
\item Graphviz - for task graph plotting
\end{itemize}

\section{Directory Structure}
The MIR source repository is structured as shown below. Directories are shown using /. Nesting is shown using indentation.
\begin{enumerate}[/]
\item src - core source files
    \begin{enumerate}[/]
        \item arch - architecture-based source files
        \item scheduling - scheduling policy source files
    \end{enumerate}
\item scripts - various profiling and testing scripts
    \begin{enumerate}[/]
        \item task-graph - scripts for task-based profiling
    \end{enumerate}
\item test - programs for testing 
    \begin{enumerate}[/]
        \item omp - OpenMP programs
        \item bots - BOTS programs written using MIR programming interface
        \item with-data-footprint - Programs where tasks have explicit data footprint 
    \end{enumerate}
\end{enumerate}

\section{Building}
Building the basic MIR library is simple. Follow below steps:
\begin{enumerate}
\item Set MIR\_ROOT environment variable to the MIR source repository path
\begin{lstlisting}[style=BashInputStyle]
$ export MIR_ROOT=<<MIR source repository path>>
\end{lstlisting}
\item Ensure MIR\_ROOT/src/SConstruct matches your build intention
\item Build
\begin{lstlisting}[style=BashInputStyle]
$ cd $MIR_ROOT/src
$ scons 
\end{lstlisting}
\end{enumerate}

\subsection{Data-distribution and locality-aware scheduling on NUMA systems}
Follow below steps to enable support for NUMA systems.
\begin{enumerate}
\item Install libnuma and numactl
\item Create an empty file called HAVE\_LIBNUMA
\begin{lstlisting}[style=BashInputStyle]
$ touch $MIR_ROOT/src/HAVE_LIBNUMA
\end{lstlisting}
\item Clean and rebuild MIR
\begin{lstlisting}[style=BashInputStyle]
$ cd $MIR_ROOT/src
$ scons -c
$ scons 
\end{lstlisting}
\end{enumerate}

\section{Task-based Programming Interface}
MIR provides a basic interface for task-based programming. A simple example program is shown below. Look at MIR\_ROOT/src/mir\_public\_int.h for interface details.

\begin{lstlisting}[style=BashInputStyle]
#include "mir_public_int.h"

void foo(int id)
{
    printf(stderr, "Hello from task %d\n", id);
}

struct foo_wrapper_arg_t 
{
    int id;
};

void foo_wrapper(void* arg)
{
    struct foo_wrapper_arg_t* farg = (struct foo_wrapper_arg_t*)(arg);
    foo(farg->id);
}

int main(int argc, char *argv[])
{
    // Initialize the runtime system
    mir_create();

    // Create as many tasks as there are threads
    int num_workers = mir_get_num_threads();
    for(int i=0; i<num_workers; i++)
    {
            struct foo_wrapper_arg_t arg;
            arg.id = i;
            mir_task_create((mir_tfunc_t) foo_wrapper, &arg, sizeof(struct foo_wrapper_arg_t), 0, NULL, NULL);
    }
        
    // Wait for tasks to finish
    mir_task_wait();

    // Release runtime system resources
    mir_destroy();

    return 0;
}
\end{lstlisting}

MIR also supports a subset of OpenMP 3.0 tasks. 

\begin{lstlisting}[style=BashInputStyle]
int main(int argc, char *argv[])
{
    // Initialize the runtime system
    mir_create();

    // Create as many tasks as there are threads
    int num_workers = mir_get_num_threads();
    for(int i=0; i<num_workers; i++)
    {
        #pragma omp task firstprivate(i)
            foo(i);
    }
        
    // Wait for tasks to finish
#pragma omp taskwait

    // Release runtime system resources
    mir_destroy();

    return 0;
}
\end{lstlisting}

Look at test programs in MIR\_ROOT/test for advanced usage examples.

\section{Testing}
MIR does not have a dedicated test case (unit testing) suite yet. The fibonacci test program is recommended for testing.

\begin{lstlisting}[style=BashInputStyle]
$ cd $MIR_ROOT/test/fib
$ scons -c
$ scons
$ echo "Executing verbose build"
$ ./fib-verbose
$ echo "Executing debug build"
$ ./fib-debug
$ echo "Executing optimized (production) build"
$ ./fib-opt
\end{lstlisting}

Other test programs in MIR\_ROOT/test can also be used for testing.

\subsection{Compiling and linking}
Look at SConstruct - the scons build file - of each test program to understand how to compile and link with the MIR library. Observing scons build messages is also recommended.

\subsection{Configuration}
MIR has several configurable options which can be set using the environment variable MIR\_CONF. Set the -h flag to see available configuration options.

\begin{lstlisting}[style=BashInputStyle]
$ cd $MIR_ROOT/test/fib
$ scons 
$ MIR_CONF="-h" ./fib-opt 3
\end{lstlisting}

\section{Profiling}
MIR supports detailed thread-based and task-based profiling.

\subsection{Thread-based profiling}
Thread states and events are main performance indicators in thread-based profiling. 

Enable the -r flag to get detailed per-thread state and event information in files with .rec extension.
Each rec file represents a worker thread.
The name of each rec file begins with the Unix time when the runtime system was initialized.
The rec files can be inspected individually or combined and visualized using Paraver.

\begin{lstlisting}[style=BashInputStyle]
$ rm *.rec *.prv *.pcf
$ MIR_CONF="-r" ./fib-opt
$ $MIR_ROOT/scripts/mirtoparaver.py *-config.rec 
$ wxparaver *-paraver.prv
\end{lstlisting}

A set of files matching the pattern <<Unix time>>-state-time*.rec are also created when -r is enabled. 
These files contain thread state duration information which can be aggregated for analysis without Paraver.

\begin{lstlisting}[style=BashInputStyle]
$ $MIR_ROOT/scripts/get-state-stats.sh <<Unix time>>
$ cat state-file-acc.info
\end{lstlisting}

Hardware performance counters can be read during thread events. This process is not fully automated and needs work from the user.

First install PAPI. 
Then set the PAPI\_ROOT environment variable and create a file called HAVE\_PAPI in MIR\_ROOT/src.
Next enable the preprocessor definition MIR\_RECORDER\_USE\_HW\_PERF\_COUNTERS in MIR\_ROOT/src/mir\_defines.h.
Next enable hardware performance counters of interest in MIR\_ROOT/src/mir\_recorder.{c|h}.
Rebuild MIR.

\begin{lstlisting}[style=BashInputStyle]
$ export PAPI_ROOT=<<PAPI install path>>
$ touch $MIR_ROOT/src/HAVE_PAPI
$ cat mir_defines.h mir_recorder.{c|h}
$ scons
\end{lstlisting}

Performance counter readings will be now be added to .rec files produced by enabling thread-based profiling (-r flag). 
The readings can either be viewed on Paraver or aggregated for analysis outside Paraver. 

\begin{lstlisting}[style=BashInputStyle]
$ $MIR_ROOT/scripts/get-event-counts.sh <<.prv file>>
$ cat event-counts-*.txt
\end{lstlisting}

\subsection{Task-based profiling}
Task are first-class citizens in task-based profiling.

Enable the -i flag in MIR\_CONF to get basic task-based information in a file called mir-stats.

\begin{lstlisting}[style=BashInputStyle]
$ MIR_CONF="-i" ./fib-opt
$ cat mir-stats
\end{lstlisting}

Enable the -g flag to generate and plot the fork-join task graph that unfolded during execution.

\begin{lstlisting}[style=BashInputStyle]
$ MIR_CONF="-g" ./fib-opt
$ Rscript ${MIR_ROOT}/scripts/task-graph/mir-fork-join-graph-plot.R mir-task-graph color
\end{lstlisting}

To generate an instruction-level profile of tasks, first get PIN sources and set the below environment variables.
\begin{lstlisting}[style=BashInputStyle]
$ export PIN_ROOT=<<Pin source path>>
$ export LD_LIBRARY_PATH=$LD_LIBRARY_PATH:<<Pin install path>>
\end{lstlisting}

Next, edit PIN\_ROOT/source/tools/Config/makefile.unix.config and add -fopenmp (note the -) to variables 
\begin{tiny}{TOOL\_LDFLAGS\_NOOPT, TOOL\_CXXFLAGS\_NOOPT}.\end{tiny}

Next build mir\_outline\_function\_profiler.so - the Pin tool which profiles tasks.
\begin{lstlisting}[style=BashInputStyle]
$ cd $MIR_ROOT/scripts/task-graph
$ make PIN_ROOT=$PIN_ROOT
\end{lstlisting}

View profiling options options using -h.
\begin{lstlisting}[style=BashInputStyle]
$ $PIN_ROOT/intel64/bin/pinbin -t $MIR_ROOT/scripts/task-graph/obj-intel64/mir_outline_function_profiler.so -h -- echo
\end{lstlisting}

Look at MIR\_ROOT/docs/ATG.pdf for more information on instruction-level profiling.
Also, the file MIR\_ROOT/test/fib/profile-test.sh shows how to automate instruction-level profiling.
\end{document}
