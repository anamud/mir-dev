\documentclass[11pt,a4paper,notitlepage]{article}
\usepackage[utf8]{inputenc}
\usepackage[T1]{fontenc}
\usepackage{graphicx}
\usepackage{enumerate}
\usepackage{xcolor}
\definecolor{bg}{rgb}{0.95,0.95,0.95}
\usepackage{ulem}
\usepackage{listings}
\usepackage{vhistory}

\lstdefinestyle{BashInputStyle}{
  language=bash,
  basicstyle=\small\sffamily,
  commentstyle=\color{black},    
  numberstyle=\tiny\color{black}, 
  keywordstyle=\color{black},       
  extendedchars=true,              
  numbers=left,
  numbersep=3pt,
  frame=tb,
  columns=fullflexible,
  backgroundcolor=\color{yellow!20},
  linewidth=\linewidth,
  breaklines=true,
  breakatwhitespace=false,           
  showspaces=false,
  keepspaces=true,                 
  captionpos=b,                    
  showspaces=false,                
  showstringspaces=false,          
  showtabs=false,                  
  tabsize=2,                       
}

\author{Ananya Muddukrishna}
\date{ananya@kth.se}
\title{Annotated Task Graph (ATG)}

\begin{document}
\maketitle

% Start of the revision history table
\begin{versionhistory}
\vhEntry{1.0}{2014-06-19}{ananya} {Created}
\end{versionhistory}

\section{Getting the ATG}

\begin{lstlisting}[style=BashInputStyle]
$ cd $MIR_ROOT/test/fib

$ echo "Examining executable for names of outline and callable functions ..."
$ $MIR_ROOT/scripts/task-graph/profiler_params.py prof-build/*.o
Using "._omp_fn.|ol_" as outline function name pattern
Processing file: prof-build/fib.o
OUTLINE_FUNCTIONS=ol_fib_0,ol_fib_1,ol_fib_2
CALLABLE_FUNCTIONS=fib_seq,fib,get_usecs,main

$ LD_LIBRARY_PATH=$LD_LIBRARY_PATH:$PIN_ROOT/intel64/runtime \
    MIR_CONF="-w=1 -g -p" \
    $PIN_ROOT/intel64/bin/pinbin \
    -t ${MIR_ROOT}/scripts/task-graph/obj-intel64/mir_outline_function_profiler.so \
    -o fib_test \
    -s ol_fib_0,ol_fib_1,ol_fib_2 \
    -c fib,fib_seq \
    -- ./fib-prof 10 4

$ mv mir-task-graph fib_test-fork_join_task_graph

$ echo "Summarizing fork join task graph ..."
$ Rscript ${MIR_ROOT}/scripts/task-graph/mir-fork-join-graph-info.R fib_test-fork_join_task_graph 

$ echo "Plotting fork join task graph ..."
$ Rscript ${MIR_ROOT}/scripts/task-graph/mir-fork-join-graph-plot.R fib_test-fork_join_task_graph color

$ echo "Annotating fork join task graph ..."
$ Rscript ${MIR_ROOT}/scripts/task-graph/mir-annotate-graph.R fib_test-fork_join_task_graph fib_test-call_graph fib_test

$ echo "Plotting annotated task graph ..."
$ Rscript ${MIR_ROOT}/scripts/task-graph/mir-annotated-graph-plot.R fib_test-annotated_task_graph color

$ echo "Listing ATG files ..."
$ ls fib_test*
\end{lstlisting}

\subsection{ATG files}
See Table~\ref{tab:atg-files}.

\begin{table}[!htb]
\begin{tabular}{|l|p{7cm}|}
\hline
\textbf{File name} & \textbf{Description} \\ \hline
fib\_test-call\_graph & Instruction-level information of tasks \\ \hline
fib\_test-mem\_map & Memory map of program execution  \\ \hline
fib\_test-fork\_join\_task\_graph & Parent-child task relationship and tgpid information \\ \hline
fib\_test-annotated\_task\_graph & Raw format of the ATG combining instruction-level and parent-child information \\ \hline
fib\_test-annotated\_task\_graph.adjm & Adjacent matrix representation of the visual format of ATG \\ \hline
fib\_test-annotated\_task\_graph.dot & Dot representation of the visual format of ATG \\ \hline
fib\_test-annotated\_task\_graph.edgelist & Edgelist representation of the visual format of ATG \\ \hline
fib\_test-annotated\_task\_graph.graphml & GraphML representation of the visual format of ATG \\ \hline
fib\_test-annotated\_task\_graph.info & Summary information about visual format of the ATG. Includes work, span and critical path from Cilk theory. \\ \hline
fib\_test-fork\_join\_task\_graph.adjm & Adjacent matrix representation of the visual format of ATG without instruction-level information \\ \hline
fib\_test-fork\_join\_task\_graph.dot & Dot representation of the visual format of ATG without instruction-level information \\ \hline
fib\_test-fork\_join\_task\_graph.edgelist & Edgelist representation of the visual format of ATG without instruction-level information \\ \hline
fib\_test-fork\_join\_task\_graph.graphml & GraphML representation of the visual format of ATG without instruction-level information \\ \hline
fib\_test-fork\_join\_task\_graph.info & Summary information about visual format of ATG without instruction-level information. Includes number of tasks and join degree distribution. \\ \hline
\end{tabular}
\caption{ATG files}
\label{tab:atg-files}
\end{table}

\section{Raw format of the ATG}
The ATG raw format is a csv file.

\begin{lstlisting}[style=BashInputStyle]
$ head fib_test-annotated_task_graph
"task","parent","joins_at","tgpid","ins_count","stack_read","stack_write","mem_fp","ccr","clr","mem_read","mem_write","name"
1,0,0,"0.",59,11,15,5,12,15,4,1,"ol_fib_2"
2,1,0,"1.",60,10,15,5,12,15,4,1,"ol_fib_0"
3,1,0,"2.",60,10,15,5,12,15,4,1,"ol_fib_1"
4,3,0,"1.2.",60,10,15,5,12,15,4,1,"ol_fib_0"
5,3,0,"2.2.",60,10,15,5,12,15,4,1,"ol_fib_1"
6,5,0,"1.2.2.",60,10,15,5,12,15,4,1,"ol_fib_0"
7,5,0,"2.2.2.",60,10,15,5,12,15,4,1,"ol_fib_1"
8,7,0,"1.2.2.2.",68,15,15,5,14,17,4,1,"ol_fib_0"
9,7,0,"2.2.2.2.",47,10,10,5,9,12,4,1,"ol_fib_1"
\end{lstlisting}

Each line shows properties of an explicit task executed by the program.
The first line shows names of the properties. 
Properties are also called annotations.
See Table~\ref{tab:raw-format}.

\begin{table}[!htb]
\begin{tabular}{|p{2cm}|p{10cm}|}
\hline
\textbf{Field} & \textbf{Description} \\ \hline
task & Identifier of the task \\ \hline
parent & Identifier of the parent task of the task \\ \hline
joins\_at & Indicates at which call to taskwait in the parent the task synchronized. Example: 0 indicates the task synchronized with the first call to taskwait in the parent. Several children can synchronize at the same call.  \\ \hline
tgpid & Indicates the task graph position identifier. See details below.  \\ \hline
ins\_count & Indicates total number of instructions executed by the task. Profiling parameters indicate which instructions to count. Typically, instructions part of runtime system calls are excluded and calls to statically-linked functions are included.  \\ \hline
stack\_read & Indicates number of read accesses to the stack while executing instructions  \\ \hline
stack\_write & Indicates number of write accesses to the stack while executing instructions \\ \hline
ccr & Computation to Communication ratio. Indicates number of instructions executed per read or write access to memory  \\ \hline
clr & Computation to Load ratio. Indicates number of instructions executed per read access to memory  \\ \hline
mem\_read & Indicates number of read accesses to memory (excluding stack) while executing instructions  \\ \hline
mem\_write & Indicates number of write accesses to memory (excluding stack) while executing instructions  \\ \hline
name & Indicates name of the outline function of the task \\ \hline
\end{tabular}
\caption{Raw format fields}
\label{tab:raw-format}
\end{table}

\subsection{Property tgpid}
The tgpid uniquely identifies a task irrespective of single-thread or many-thread execution. 
The format of tgpid is A.B.C.....

\begin{itemize}
\item 0. Represents the first task created. This is a special meaning.
\item A. means Ath child of the first task
\item A.B. means Ath child of task B.
\item A.B.C. means Ath child of task B.C.
\end{itemize}

NOTE: The tgpid is an experimental feature, not fully tested and subject to change.

\section{Visual format of the ATG}
The visual format gives shape to the raw format of the ATG.
It describes task-based execution in an intuitive manner allowing the programmer to spot performance problems.
The visual format can be viewed using graph visualization tools such as dot, yEd and cytoscape.
See Figure~\ref{fig:yed-atg} for visualization of the ATG on yEd and Figure~\ref{fig:dot-atg} for visualization of the ATG on Dot.

\begin{lstlisting}[style=BashInputStyle]
$ echo "Visualizing annotated task graph ..."
$ dot -Tpng fib_test-annotated_task_graph.dot >  fib_test-annotated_task_graph.png
$ yed fib_test-annotated_task_graph.dot 
\end{lstlisting}

\begin{figure}[!ht]
\centering
\includegraphics[width=\textwidth]{figures/fib_test-10_4-annotated_task_graph.pdf}
\caption{fib\_test-annotated\_task\_graph.graphml viewed on yEd}
\label{fig:yed-atg}
\end{figure}

\begin{figure}[!ht]
\centering
\includegraphics[width=\textwidth]{figures/fib_test-10_4-annotated_task_graph.png}
\caption{fib\_test-annotated\_task\_graph.graphml visualized using Dot}
\label{fig:dot-atg}
\end{figure}

\end{document}
